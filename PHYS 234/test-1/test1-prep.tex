\documentclass[11pt, letterpaper]{article}
\usepackage[utf8]{inputenc}
\usepackage{geometry}
\usepackage{amsmath}
\usepackage{amssymb}
\usepackage{physics} % Critical for Dirac notation \bra \ket
\usepackage{xcolor}
\usepackage{tcolorbox}
\usepackage{hyperref}

% Page Geometry
\geometry{margin=1in}

% Custom Colors
\definecolor{highlight}{RGB}{240, 248, 255}
\definecolor{border}{RGB}{70, 130, 180}

% Custom Box for Definitions
\newtcolorbox{defnbox}[1]{colback=highlight, colframe=border, title={\textbf{#1}}}

\title{\textbf{Quantum Mechanics Study Guide}\\ \large Lectures 6 -- 9: Operators, Projections, and Uncertainty}
\author{Based on Lectures by R. Budakian}
\date{Winter 2026}

\begin{document}

\maketitle
\tableofcontents
\vspace{1cm}
\hrule
\vspace{1cm}

\section{Lecture 6: Generalized Quantum Systems}

\subsection{Core Concepts}
\begin{itemize}
    \item \textbf{Postulate 2:} Physical observables are represented by \textbf{operators} acting on kets.
    \item \textbf{Postulate 3:} Measurement outcomes are the \textbf{eigenvalues} $\{a_n\}$ of that operator.
    \item \textbf{Hermitian Operators:} Observables must be Hermitian ($\hat{A} = \hat{A}^\dagger$) because:
    \begin{enumerate}
        \item They have \textbf{real eigenvalues} (measurement results must be real).
        \item Eigenvectors with distinct eigenvalues are \textbf{orthogonal}.
    \end{enumerate}
\end{itemize}

\subsection{Matrix Representation (How to Build Operators)}
To represent an operator $\hat{A}$ as a matrix in a specific basis $\{|a_n\rangle\}$, calculate the matrix elements $A_{nm}$:
\begin{equation}
    A_{nm} = \mel{a_n}{\hat{A}}{a_m}
\end{equation}
If the basis consists of eigenvectors of $\hat{A}$, the matrix is \textbf{diagonal} with eigenvalues on the diagonal.

\begin{defnbox}{Memorize: Spin Matrices (in Z-basis)}
For a spin-1/2 particle, the operators in the $S_z$ basis are:
\begin{align*}
    S_z &= \frac{\hbar}{2} \begin{pmatrix} 1 & 0 \\ 0 & -1 \end{pmatrix} \\
    S_x &= \frac{\hbar}{2} \begin{pmatrix} 0 & 1 \\ 1 & 0 \end{pmatrix} \\
    S_y &= \frac{\hbar}{2} \begin{pmatrix} 0 & -i \\ i & 0 \end{pmatrix}
\end{align*}
\end{defnbox}

\section{Lecture 7: Projection Operators}

\subsection{The Projection Operator}
A projector isolates the component of a vector along a specific direction (eigenstate).
\begin{equation}
    \hat{P}_n \equiv \ket{a_n}\bra{a_n}
\end{equation}

\subsection{Important Identities}
\begin{itemize}
    \item \textbf{Completeness Relation:} Summing all projectors returns the identity matrix. Use this to insert "1" into equations to simplify them.
    \begin{equation}
        \sum_n \hat{P}_n = \sum_n \ket{a_n}\bra{a_n} = \mathbb{I}
    \end{equation}
    \item \textbf{Spectral Decomposition:} Any operator can be written as a sum of its eigenvalues weighted by its projectors.
    \begin{equation}
        \hat{A} = \sum_n a_n \hat{P}_n = \sum_n a_n \ket{a_n}\bra{a_n}
    \end{equation}
\end{itemize}

\subsection{Postulate 5: Measurement Collapse}
If you measure observable $\hat{A}$ and get result $a_n$, the state collapses.
\begin{defnbox}{Collapse Formula}
\begin{equation}
    \ket{\psi_f} = \frac{\hat{P}_n \ket{\psi_i}}{\sqrt{\ev{\hat{P}_n}{\psi_i}}}
\end{equation}
\end{defnbox}

\subsection{Solving Stern-Gerlach (S-G) Problems}
\textbf{Problem Type:} "What is the probability of exiting the final port?" or "What is the final state?"

\textbf{Step-by-Step Method:}
\begin{enumerate}
    \item \textbf{Identify the Input:} Start with the initial state vector $\ket{\psi_{in}}$.
    \item \textbf{Apply Projectors:} For every S-G analyzer the particle passes through, apply the corresponding projection operator.
    \begin{itemize}
        \item Example: Passing through an $S_z+$ filter means applying $\hat{P}_{z+} = \ket{+}\bra{+}$.
        \item $\ket{\psi_{after}} = \hat{P}_{filter} \ket{\psi_{before}}$.
    \end{itemize}
    \item \textbf{Normalize (if asking for state):} If asking for the state after measurement, divide by the norm (length) of the vector.
    \item \textbf{Calculate Probability (if asking for prob):}
    \begin{equation}
        P = |\braket{\text{final state}}{\text{current state}}|^2 \quad \text{OR} \quad P = \ev{\hat{P}_{final}}{\psi_{current}}
    \end{equation}
\end{enumerate}

\section{Lecture 8: Expectation Values}

\subsection{Definitions}
The \textbf{Expectation Value} is the average result of a large number of measurements on identical systems.
\begin{equation}
    \ev{\hat{A}} = \ev{\hat{A}}{\psi}
\end{equation}

The \textbf{RMS Deviation} (Uncertainty) is:
\begin{equation}
    \Delta A = \sqrt{\ev{A^2} - \ev{A}^2}
\end{equation}

\subsection{Calculation Strategy}
To calculate $\Delta A$:
\begin{enumerate}
    \item Calculate $\ev{A}$:
    \[ \ev{A} = \mel{\psi}{\hat{A}}{\psi} \]
    (Matrix multiplication: Row Vector $\times$ Matrix $\times$ Column Vector).
    \item Calculate $\hat{A}^2$: Square the matrix $\hat{A}$.
    \item Calculate $\ev{A^2}$:
    \[ \ev{A^2} = \mel{\psi}{\hat{A}^2}{\psi} \]
    \item Plug into the RMS formula.
\end{enumerate}

\section{Lecture 9: The Uncertainty Principle}

\subsection{Commutators}
The commutator measures if two operators commute (order doesn't matter) or not.
\begin{equation}
    [\hat{A}, \hat{B}] \equiv \hat{A}\hat{B} - \hat{B}\hat{A}
\end{equation}
\begin{itemize}
    \item If $[\hat{A}, \hat{B}] = 0$: The observables are \textbf{compatible}. They share a common basis and can be measured simultaneously with arbitrary precision.
    \item If $[\hat{A}, \hat{B}] \neq 0$: The observables are \textbf{incompatible}.
\end{itemize}

\subsection{Spin Commutation Relations}
Remember the cyclic permutation ($x \to y \to z \to x$):
\begin{align*}
    [S_x, S_y] &= i\hbar S_z \\
    [S_y, S_z] &= i\hbar S_x \\
    [S_z, S_x] &= i\hbar S_y
\end{align*}

\subsection{Heisenberg Uncertainty Principle}
The general form for any two operators:
\begin{defnbox}{General Uncertainty Relation}
\begin{equation}
    \Delta A \Delta B \ge \frac{1}{2} |\ev{[\hat{A}, \hat{B}]}|
\end{equation}
\end{defnbox}

\subsection{Useful Identity: Magnitude of Spin}
While individual components are uncertain, the \textit{magnitude} of the spin vector is constant:
\begin{equation}
    \hat{S}^2 = \hat{S}_x^2 + \hat{S}_y^2 + \hat{S}_z^2 = \frac{3\hbar^2}{4} \mathbb{I}
\end{equation}

\end{document}