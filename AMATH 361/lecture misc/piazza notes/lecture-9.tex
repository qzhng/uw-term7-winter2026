\documentclass[12pt]{article}
\usepackage[letterpaper, margin=0.8in]{geometry}

% PACKAGES
\usepackage{adjustbox}
\usepackage{amsmath, amssymb, amsthm}
\usepackage{aliascnt}
\usepackage{bm}
\usepackage{braket}
\usepackage{empheq}
\usepackage{enumitem}
\usepackage{esint}
\usepackage{esvect}
\usepackage{graphicx}
\usepackage{mathtools}
\usepackage{hyperref}
\usepackage{cleveref} % must be included after hyperref
\usepackage{siunitx}
\usepackage{tikz}
\usetikzlibrary{patterns, arrows.meta, calc, angles, quotes, decorations.pathreplacing, decorations.markings, positioning}
\usepackage[most]{tcolorbox}
\usepackage{pgfplots}
\usepgfplotslibrary{groupplots}
\pgfplotsset{compat=1.18}

% STATEMENT ENVIRONMENT
\newtheoremstyle{conditionalstyle}
  {3pt} % Space above
  {3pt} % Space below
  {\normalfont} % Body font - regular upright
  {} % Indent amount
  {\bfseries} % Theorem head font (only used when no optional argument)
  {.} % Punctuation after theorem head
  {.5em} % Space after theorem head
  {\thmnumber{\textbf{#1 #2}}\thmnote{\normalfont\textit{ (#3)}}} % Theorem head spec
\theoremstyle{conditionalstyle}
\newtheorem{definition}{Definition}[section]

% ALIAS FOR SHARED NUMBERING
\newaliascnt{axiom}{definition}
\newtheorem{axiom}[axiom]{Axiom}
\aliascntresetthe{axiom}

\newaliascnt{lemma}{definition}
\newtheorem{lemma}[lemma]{Lemma}
\aliascntresetthe{lemma}

\newaliascnt{theorem}{definition}
\newtheorem{theorem}[theorem]{Theorem}
\aliascntresetthe{theorem}

\newaliascnt{corollary}{definition}
\newtheorem{corollary}[corollary]{Corollary}
\aliascntresetthe{corollary}

\newaliascnt{note}{definition}
\newtheorem{note}[note]{Note}
\aliascntresetthe{note}

\newaliascnt{fact}{definition}
\newtheorem{fact}[fact]{Fact}
\aliascntresetthe{fact}

\newaliascnt{example}{definition}
\newtheorem{example}[example]{Example}
\aliascntresetthe{example}

% TCOLORBOX SETUP
\tcolorboxenvironment{definition}{
  breakable,
  enhanced,
  colback=teal!5!white,
  frame hidden,
  boxrule=0pt,
  arc=0pt, outer arc=0pt,
  left=5pt, % Padding so text doesn't touch the bar
  overlay={
    \draw[teal!75!black, line width=2pt] (frame.north west) -- (frame.south west);
  },
  before skip=10pt,
  after skip=10pt
}
\tcolorboxenvironment{axiom}{
  breakable, enhanced, colback=teal!5!white, frame hidden, boxrule=0pt,
  arc=0pt, outer arc=0pt, left=5pt,
  overlay={\draw[teal!75!black, line width=2pt] (frame.north west) -- (frame.south west);},
  before skip=10pt, after skip=10pt
}
\tcolorboxenvironment{theorem}{
  breakable, enhanced,
  colback=violet!5!white,
  frame hidden, boxrule=0pt,
  arc=0pt, outer arc=0pt,
  left=5pt,
  overlay={
    \draw[violet!75!black, line width=2pt] (frame.north west) -- (frame.south west);
  },
  before skip=10pt, after skip=10pt
}
\tcolorboxenvironment{lemma}{
  breakable, enhanced, colback=violet!5!white, frame hidden, boxrule=0pt,
  arc=0pt, outer arc=0pt, left=5pt,
  overlay={\draw[violet!75!black, line width=2pt] (frame.north west) -- (frame.south west);},
  before skip=10pt, after skip=10pt
}
\tcolorboxenvironment{corollary}{
  breakable, enhanced, colback=violet!5!white, frame hidden, boxrule=0pt,
  arc=0pt, outer arc=0pt, left=5pt,
  overlay={\draw[violet!75!black, line width=2pt] (frame.north west) -- (frame.south west);},
  before skip=10pt, after skip=10pt
}
\tcolorboxenvironment{fact}{
  breakable, enhanced, colback=violet!5!white, frame hidden, boxrule=0pt,
  arc=0pt, outer arc=0pt, left=5pt,
  overlay={\draw[violet!75!black, line width=2pt] (frame.north west) -- (frame.south west);},
  before skip=10pt, after skip=10pt
}
\tcolorboxenvironment{example}{
  breakable, enhanced,
  colback=gray!5!white,
  frame hidden, boxrule=0pt,
  arc=0pt, outer arc=0pt,
  left=5pt,
  overlay={
    \draw[gray!60!black, line width=2pt] (frame.north west) -- (frame.south west);
  },
  before skip=10pt, after skip=10pt
}
\tcolorboxenvironment{note}{
  breakable, enhanced,
  colback=orange!5!white,
  frame hidden, boxrule=0pt,
  arc=0pt, outer arc=0pt,
  left=5pt,
  overlay={
    \draw[orange!80!black, line width=2pt] (frame.north west) -- (frame.south west);
  },
  before skip=10pt, after skip=10pt
}

% CLEVEREF ALIAS
\crefname{definition}{definition}{definitions}
\crefname{axiom}{axiom}{axioms}
\crefname{lemma}{lemma}{lemmas}
\crefname{theorem}{theorem}{theorems}
\crefname{corollary}{corollary}{corollaries}
\crefname{note}{note}{notes}
\crefname{fact}{fact}{facts}
\crefname{example}{example}{examples}

\crefalias{axiom}{axiom}
\crefalias{lemma}{lemma}
\crefalias{theorem}{theorem}
\crefalias{corollary}{corollary}
\crefalias{note}{note}
\crefalias{fact}{fact}
\crefalias{example}{example}

\Crefname{definition}{Definition}{Definitions}
\Crefname{axiom}{Axiom}{Axioms}
\Crefname{lemma}{Lemma}{Lemmas}
\Crefname{theorem}{Theorem}{Theorems}
\Crefname{corollary}{Corollary}{Corollaries}
\Crefname{note}{Note}{Notes}
\Crefname{fact}{Fact}{Facts}
\Crefname{example}{Example}{Examples}
\Crefname{equation}{Eq.}{Eqs.}

% BRACKETS TYPESET
\newcommand{\lp}{\left(}
\newcommand{\rp}{\right)}
\newcommand{\lb}{\left[}
\newcommand{\rb}{\right]}
\newcommand{\lc}{\left\{}
\newcommand{\rc}{\right\}}
\newcommand{\lv}{\lvert}
\newcommand{\rv}{\rvert}
\newcommand{\lV}{\lVert}
\newcommand{\rV}{\rVert}

% DELIMITER
\DeclarePairedDelimiter{\abs}{\lvert}{\rvert}
\DeclarePairedDelimiter{\norm}{\lVert}{\rVert}
\DeclarePairedDelimiter{\inner}{\langle}{\rangle}
\DeclarePairedDelimiter{\floor}{\lfloor}{\rfloor}
\DeclarePairedDelimiter{\ceil}{\lceil}{\rceil}

% SET SPACE
\usepackage{setspace}
\onehalfspacing

% ---------- DOCUMENT ----------
\begin{document}

\section*{Lecture 9}
\addcontentsline{toc}{section}{Lecture 9}
\stepcounter{section}
\setcounter{section}{9}
\setcounter{equation}{0}

\subsection{Surface Forces and Pressure}

Last time, using Newton's 2nd Law and including gravity (volume force) and surface forces, we obtained the integral form of the momentum equation:
\[
\iiint_{W(t)} \rho \frac{D u_i}{Dt} \, dV = - \iiint_{W(t)} \rho \frac{\partial \Pi}{\partial x_i} \, dV + \oiint_{\partial W(t)} t_i \, dS.
\]

Today, we focus on the surface force \& rewrite it as a triple integral over \(W(t)\).

\begin{center}
\begin{tikzpicture}[scale=1, line cap=round, line join=round]
    % Sphere Outline
    \draw[thick] (0,0) circle (1.2);
    
    % Equator (3D effect)
    \draw[dashed] (1.2,0) arc (0:180:1.2 and 0.4);
    \draw (1.2,0) arc (0:-180:1.2 and 0.4);
    
    % Label W(t) inside
    \node at (-2,0.3) {\small $W(t)$};
    
    % Surface patch (small square on surface)
    \begin{scope}[rotate=45]
        \draw[fill=white] (0.8,-0.15) rectangle (1.1,0.15);
        % Normal vector n-hat
        \draw[->, thick] (0.95, 0) -- (1.8, 0) node[right] {$\hat{\bm n}$};
    \end{scope}

    % Labels to the right of the sphere
    \node[align=left, anchor=west] at (2, 0.2) {$\partial W(t) = \text{boundary}$};
    \node[align=left, anchor=west] at (2, -0.4) {$d\bm f = \bm t \, ds$};
\end{tikzpicture}
\end{center}

On each subsurface, we have
\[
d\bm f = \bm t \, ds
\]
where \(\bm t\) is the stress vector. Let's consider a state of equilibrium where there is no movement
\[
\bm u = \vec{0}.
\]

\subsubsection{The Stress Vector in Equilibrium}

Consider a state of equilibrium where there is no movement (\(\bm u = \vec{0}\)).
Recall that fluids in equilibrium cannot resist shear forces (stresses). Since we have \(\bm u = \vec{0}\), in equilibrium, the surface force must be parallel to the normal vector \(\hat{\bm n}\). Then the stress tensor is
\[
\bm t = - p(\bm x) \hat{\bm n},
\]
where we call \(p(\bm x)\) the pressure. The negative sign indicates that pressure acts \emph{inward} on the volume (compression).

Our global expression for the surface force becomes
\[
\bm F_s = \oiint_{\partial W(t)} - p \hat{\bm n} \, dS.
\]

\subsubsection{Derivation of Pressure Force}

To rewrite this surface integral as a volume integral, we consider two approaches: a heuristic Taylor expansion and a formal proof using Gauss' Theorem.

\begin{note}[Heuristic Derivation using Taylor Expansion]
    Consider the pressure on a small material volume (a cube) with side lengths \(dx, dy, dz\) such that \(dV = dx \, dy \, dz \ll 1\).
    
    \begin{center}
        \begin{tikzpicture}[
        x={(-0.5cm,-0.4cm)}, y={(1cm,0cm)}, z={(0cm,1cm)}, % Define 3D axis orientation
        scale=2,
        >=Stealth
        ]

        % --- Definitions ---
        \def\s{1.2} % Cube side length

        % --- Axes ---
        % Draw axes starting from the origin (back corner of the cube)
        \draw[->] (0,0,0) -- (\s+0.8,0,0) node[below left] {$x$};
        \draw[->] (0,0,0) -- (0,\s+0.8,0) node[right] {$y$};
        \draw[->] (0,0,0) -- (0,0,\s+0.8) node[above] {$z$};

        % --- Cube (Visible Lines Only) ---
        % We removed the dashed lines from (0,0,0) because they overlap the axes.
        
        % Front Face (x=s)
        \draw (\s,0,0) -- (\s,\s,0) -- (\s,\s,\s) -- (\s,0,\s) -- cycle;
        
        % Top Face (z=s)
        \draw (0,0,\s) -- (\s,0,\s) -- (\s,\s,\s) -- (0,\s,\s) -- cycle;
        
        % Right Face (y=s)
        \draw (0,\s,0) -- (\s,\s,0) -- (\s,\s,\s) -- (0,\s,\s) -- cycle;
        
        % Bottom Face edges (visible ones not on axes)
        % (s,0,0) to (s,s,0) is part of Front Face
        % (0,s,0) to (s,s,0) is part of Right Face
        
        % Left Face edges (visible ones not on axes)
        % (0,0,s) to (0,s,s) is part of Top/Right boundary
        
        % --- Completing the "Wireframe" feel without the hidden back lines ---
        % We need to make sure the cube looks "anchored". 
        % The lines (s,0,0)--(s,s,0) and (0,s,0)--(s,s,0) define the bottom visible corner.
        
        % Pressure Vectors
        % Top (pushing down)
        \draw[->] (0.5*\s, 0.5*\s, \s + 0.6) -- (0.5*\s, 0.5*\s, \s);
        \node[right] at (0.5*\s, 0.5*\s, \s + 0.6) {$p(z+dz)dxdy$};
        
        % Bottom (pushing up)
        \draw[->] (0.5*\s, 0.5*\s, -0.6) -- (0.5*\s, 0.5*\s, 0);
        \node[right] at (0.5*\s, 0.5*\s, -0.6) {$p(z)dxdy$};

        \end{tikzpicture}
    \end{center}

    The units of \(\bm t\) are \([\bm t] = \text{N}/\text{m}^2\).
    We will look at the balance of forces in the \(z\)-direction. For the difference in this force from the two faces, we Taylor expand about \(z\).
    \begin{align*}
        dF_z &= p(x,y,z) \, dx dy - p(x,y,z+dz) \, dx dy \\
        &= (p(x,y,z) - p(x,y,z+dz)) \, dx dy \\
        &= (p(x,y,z) - [p(x,y,z) + dz \frac{\partial p}{\partial z}(x,y,z) + \dots]) \, dx dy \\
        &= - \frac{\partial p}{\partial z}(x,y,z) \, dx dy dz \\
        &= - \frac{\partial p}{\partial z} \, dV.
    \end{align*}
    Similarly,
    \[
    dF_x = - \frac{\partial p}{\partial x}(x,y,z) \, dx dy dz \quad \& \quad dF_y = - \frac{\partial p}{\partial y}(x,y,z) \, dx dy dz
    \]
    Combining these into a vector equation:
    \[
    d\bm F = (dF_x, dF_y, dF_z) = - \lp \frac{\partial p}{\partial x}, \frac{\partial p}{\partial y}, \frac{\partial p}{\partial z} \rp \, dV = - \nabla p \, dV.
    \]
    Integrating over the whole volume, the surface force can be written as:
    \[
    \bm F_s = - \iiint_{W(t)} \nabla p \, dV.
    \]
\end{note}

This very rough calculation suggests the following identity.

\begin{theorem}[Gradient Theorem Corollary]
    \label{pressure gradient identity}
    \[
    \oiint_{\partial W(t)} p \, \hat{\bm n} \, dS = \iiint_{W(t)} \nabla p \, dV.
    \]
\end{theorem}

\begin{proof}
    The idea of the proof is to write the identity in component form using Gauss' Divergence Theorem.
    Recall Gauss' Divergence Theorem states:
    \[
    \iiint_W \nabla \cdot \bm U \, dV = \oiint_{\partial W} \bm U \cdot \hat{\bm n} \, dS.
    \]
    We want to show:
    \[
    \oiint_{\partial W(t)} p (n_x, n_y, n_z) \, dS = \iiint_{W(t)} \lp \frac{\partial p}{\partial x}, \frac{\partial p}{\partial y}, \frac{\partial p}{\partial z} \rp \, dV.
    \]
    This corresponds to 3 scalar equations.
    
    \textbf{For the x-direction:} Pick the vector field \(\bm U = (p, 0, 0)\). Then \(\nabla \cdot \bm U = \frac{\partial p}{\partial x}\).
    Applying Gauss' Theorem:
    \[
    \iiint_{W(t)} \frac{\partial p}{\partial x} \, dV = \oiint_{\partial W(t)} (p, 0, 0) \cdot \hat{\bm n} \, dS = \oiint_{\partial W(t)} p n_x \, dS.
    \]
    \textbf{For the y-direction:} Pick \(\bm U = (0, p, 0)\). Then \(\nabla \cdot \bm U = \frac{\partial p}{\partial y}\).
    \[
    \iiint_{W(t)} \frac{\partial p}{\partial y} \, dV = \oiint_{\partial W(t)} p n_y \, dS.
    \]
    \textbf{For the z-direction:} Pick \(\bm U = (0, 0, p)\). Then \(\nabla \cdot \bm U = \frac{\partial p}{\partial z}\).
    \[
    \iiint_{W(t)} \frac{\partial p}{\partial z} \, dV = \oiint_{\partial W(t)} p n_z \, dS.
    \]
    Combining these 3 scalar equations into a vector equation yields the result.
\end{proof}

\subsection{Hydrostatics}

Given this identity, we return to Newton's 2nd Law in a state of rest (\(\bm u = \vec{0}\)). The acceleration term is zero.
\begin{align*}
    0 &= - \iiint_{W(t)} \rho \nabla \Pi \, dV - \oiint_{\partial W(t)} p \hat{\bm n} \, dS \\
      &= \iiint_{W(t)} \lb - \rho \nabla \Pi - \nabla p \rb \, dV \\
      &= - \iiint_{W(t)} \lb \rho \nabla \Pi + \nabla p \rb \, dV.
\end{align*}
Since this holds for any arbitrary volume \(W(t)\), we apply the Dubois-Reymond Lemma (localization) to get:
\[
\rho \nabla \Pi + \nabla p = 0.
\]
Assuming gravity acts in the vertical direction, \(\Pi = gz\) and \(\nabla \Pi = g \hat{\bm z}\).

\begin{definition}[Hydrostatic Balance]
    The hydrostatic balance equation is
    \[
    \nabla p = - \rho \nabla \Pi = - \rho g \hat{\bm z},
    \]
    where \(\Pi = gz\). In component form:
    \[
    \frac{\partial p}{\partial x} = 0, \quad \frac{\partial p}{\partial y} = 0, \quad \frac{\partial p}{\partial z} = - \rho g.
    \]
    This describes the perfect balance between the forces of gravity and pressure.
\end{definition}

\subsubsection{Hydrostatic Balance of the Ocean}

Suppose we consider a simple ocean at rest (\(\bm u = \bm 0\)), and the density is constant \(\rho = \rho_0\) (incompressible).
Since \(\frac{\partial p}{\partial x} = 0\) and \(\frac{\partial p}{\partial y} = 0\), pressure depends only on \(z\), i.e., \(p(z)\).

\begin{center}
\begin{tikzpicture}[>=Stealth]
    % Air/Ocean interface
    \draw[thick] (0,0) -- (6,0) node[right] {$z=0$};
    \node at (3, 0.4) {air};
    \node at (3, -0.4) {ocean};

    % Define the curve of the ocean bottom
    \def\oceanbottom{
        (0,-2.5) .. controls (1,-2.8) and (2,-1.5) .. (3,-2.2) .. controls (4,-3.0) and (5,-2.0) .. (6,-2.5)
    }

    % 1. Fill the ground with hatching (Pattern)
    % We extend the path down to -3.5 to create a closed shape for filling
    \fill[pattern=north east lines, pattern color=black!60] 
        \oceanbottom -- (6,-3.5) -- (0,-3.5) -- cycle;

    % 2. Draw the thick boundary line on top
    \draw[thick, smooth] \oceanbottom node[right] {$z = -H(x,y)$};
\end{tikzpicture}
\end{center}

The vertical equation is:
\[
\frac{\partial p}{\partial z} = - g \rho_0.
\]
We integrate from depth \(z\) to the surface \(z=0\):
\begin{align*}
    \int_z^0 \frac{dp}{dz} \, dz &= \int_z^0 - g \rho_0 \, dz \\
    p(z)\Big|_z^0 &= -g \rho_0 z \Big|_z^0 \\
    p(0) - p(z) &= 0 - (- g \rho_0 z) = g \rho_0 z \\
    p(z) &= p(0) - \rho_0 g z.
\end{align*}
If we denote the atmospheric pressure as \(P_{\text{atmosphere}} = p(0)\), then
\[
p(z) = P_{\text{atmosphere}} - \rho_0 g z.
\]

\begin{note}
    Since \(z\) is negative underwater, \(-\rho_0 g z\) is a positive term.
    The pressure at \(z\) is equal to the pressure of the atmosphere plus the weight of the fluid column above per unit area:
    \[
    \frac{\rho_0 g (-z) dA}{dA} = \frac{m g}{dA}.
    \]
    \begin{center}
        \begin{tikzpicture}[>=Stealth, scale=1.2]
            % Axes
            \draw[->, thick] (0,0) -- (4,0) node[right] {$p$};
            \draw[->, thick] (0,0) -- (0,-3) node[below] {$z$};

            % Label for z-axis direction
            % (Implicitly z is depth here based on the graph direction)

            % The Plot
            % Starts at P_atmosphere on the x-axis and increases linearly as z goes down
            \draw[thick] (1,0) -- (3.5,-2.5);

            % Intercept Label
            \draw (1,0.1) -- (1,-0.1); % tick mark
            \node[above] at (1,0.1) {$P_{\text{atmosphere}}$};

        \end{tikzpicture}
    \end{center}
\end{note}

\begin{example}[Numerical Scale]
    It is observed that \(P_{\text{atmosphere}} \approx 10^5 \, \text{N}/\text{m}^2\).
    The depth of the ocean is \(\le 10 \, \text{km} = 10^4 \, \text{m}\).
    The density of water is \(\rho_0 \approx 10^3 \, \text{kg}/\text{m}^3\) and \(g \approx 10 \, \text{m}/\text{s}^2\).
    
    At the top of the ocean: \(p \approx 10^5 \, \text{N}/\text{m}^2\).
    
    At the bottom of the ocean:
    \[
    p \approx 10^5 \, \text{N}/\text{m}^2 + \lp 10^3 \frac{\text{kg}}{\text{m}^3} \rp \lp 10 \frac{\text{m}}{\text{s}^2} \rp \lp 10^4 \, \text{m} \rp \approx 10^8 \, \text{N}/\text{m}^2.
    \]
    The pressure at the bottom is 1000 times larger than at the surface.
\end{example}

\subsubsection{Hydrostatic Balance for the Atmosphere}

The density of air changes a lot (it is compressible). To describe air, you need an equation of state. One choice is the Ideal Gas Law:
\[
p = \rho R T, \quad R \approx 287 \, \frac{\text{J}}{\text{kg}\cdot\text{K}}.
\]
The temperature is not constant, but if we assume it is for simplicity (say \(T = T_0 = \text{const}\)), we get simple equations.
From the ideal gas law: \(\rho = \frac{p}{R T_0}\).
Substitute this into the vertical hydrostatic equation:
\[
\frac{dp}{dz} = - g \rho = - \frac{g p}{R T_0}.
\]
This is a separable ODE:
\[
\frac{dp}{p} = - \frac{g}{R T_0} dz \quad \Rightarrow \quad \ln p = - \frac{g z}{R T_0} + C.
\]
Solving for \(p\):
\[
p(z) = p(0) e^{-z/H}, \quad \text{where } H \equiv \frac{R T_0}{g}.
\]
\(H\) is called the \textbf{Scale Height}. For \(T_0 \approx 20^\circ\text{C}\), \(H \approx 8.4 \, \text{km}\).

\begin{center}
\begin{tikzpicture}[scale=0.9]
    % Axes
    \draw[->] (0,0) -- (4,0) node[right] {$T$};
    \draw[->] (0,0) -- (0,4) node[above] {$z$};
    
    % Ticks
    \draw (0.1, 1) -- (-0.1, 1) node[left] {\small 0 km};
    \draw (0.1, 2) -- (-0.1, 2) node[left] {\small 10 km};
    \draw (0.1, 3.5) -- (-0.1, 3.5) node[left] {\small 50 km};
    
    % Temperature Profile (approximate)
    \draw[thick] (3, 0.5) -- (1.5, 2) -- (3, 3.5);
    \draw[dashed] (1.5, 0) -- (1.5, 2);
    
    % Labels
    \node[below] at (1.5, 0) {\small -40$^\circ$C};
    \node[below] at (3, 0) {\small 20$^\circ$C};
\end{tikzpicture}
\end{center}

\end{document}