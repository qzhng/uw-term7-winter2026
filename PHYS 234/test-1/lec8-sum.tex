\documentclass[11pt]{article}
\usepackage[utf8]{inputenc}
\usepackage{amsmath, amssymb}
\usepackage{geometry}
\usepackage{physics} % For bra-ket notation \ket{}, \bra{}, \braket{}
\usepackage{hyperref}

\geometry{a4paper, margin=1in}

\title{Lecture 8: Expectation Value \& RMS Deviation - Summary}
\author{Based on Lecture by Raffi Budakian (Univ. of Waterloo)}
\date{Week of Jan. 26, 2026}

\begin{document}

\maketitle

\section{Lecture Overview}
\begin{itemize}
    \item \textbf{Topic:} State Tomography, Expectation Values, and Uncertainty (RMS Deviation).
    \item \textbf{Goal:} To quantify the ``average'' outcome of a quantum measurement and the ``spread'' or uncertainty associated with it.
\end{itemize}

\section{1. Introduction to State Tomography}
\begin{itemize}
    \item A single measurement collapses the state, destroying the information we wanted to measure.
    \item To determine an unknown state $\ket{\psi}$, we need many identical copies.
    \item By measuring these copies repeatedly, we build a probability distribution from which we can reconstruct (tomography) the original state.
\end{itemize}

\section{2. The Expectation Value}
The expectation value $\langle \hat{A} \rangle$ is the theoretical average of many measurements.

\subsection{Definition}
It is defined as the sum of eigenvalues weighted by their probabilities. In the operator formalism:
\begin{equation}
    \langle \hat{A} \rangle = \bra{\psi} \hat{A} \ket{\psi}
\end{equation}
This quantity is physically observable and independent of the basis used to represent the vectors.

\section{3. RMS Deviation (Uncertainty)}
While the expectation value gives the mean, the Root-Mean-Square (RMS) deviation gives the spread (standard deviation) of the measurement results.

\subsection{Formula}
\begin{equation}
    \Delta A = \sqrt{\langle A^2 \rangle - \langle A \rangle^2}
\end{equation}
Where $\langle A^2 \rangle$ is the expectation value of the operator squared ($\hat{A} \cdot \hat{A}$).

\section{4. Example: Spin Measurements}
The lecture analyzes the uncertainty relations using Spin-1/2 operators.

\subsection{Properties of Squared Spin Operators}
For spin-1/2 matrices, the square of any Pauli matrix is the Identity.
\begin{equation}
    S_x^2 = S_y^2 = S_z^2 = \frac{\hbar^2}{4} \mathbb{1}
\end{equation}
Consequently, the expectation value of any squared spin component is always:
\begin{equation}
    \langle S_i^2 \rangle = \frac{\hbar^2}{4}
\end{equation}

\subsection{Case Study: Measuring $S_z$ vs $S_x$}
Assume the system is in the state $\ket{+}$ (an eigenstate of $S_z$).

\begin{enumerate}
    \item \textbf{Measuring $S_z$ (The "Same" Direction):}
    \begin{itemize}
        \item $\langle S_z \rangle = \frac{\hbar}{2}$
        \item $\Delta S_z = \sqrt{\frac{\hbar^2}{4} - (\frac{\hbar}{2})^2} = 0$
        \item \textbf{Result:} Zero uncertainty. The result is deterministic.
    \end{itemize}

    \item \textbf{Measuring $S_x$ (The "Orthogonal" Direction):}
    \begin{itemize}
        \item $\langle S_x \rangle = 0$ (Average of $+ \hbar/2$ and $- \hbar/2$)
        \item $\Delta S_x = \sqrt{\frac{\hbar^2}{4} - (0)^2} = \frac{\hbar}{2}$
        \item \textbf{Result:} Maximum uncertainty. The result is completely random.
    \end{itemize}
\end{enumerate}

\newpage

\section*{Midterm Formula Sheet}
\addcontentsline{toc}{section}{Midterm Formula Sheet}

\begin{enumerate}
    \item \textbf{Expectation Value (Operator Form)}
    \begin{equation}
        \langle \hat{A} \rangle = \bra{\psi} \hat{A} \ket{\psi}
    \end{equation}

    \item \textbf{Expectation Value (Summation Form)}
    \begin{equation}
        \langle \hat{A} \rangle = \sum_n a_n P_n
    \end{equation}

    \item \textbf{RMS Deviation (Uncertainty)}
    \begin{equation}
        \Delta A = \sqrt{\langle A^2 \rangle - \langle A \rangle^2}
    \end{equation}

    \item \textbf{Squared Spin Operator Identity}
    \begin{equation}
        S_i^2 = \frac{\hbar^2}{4} \mathbb{1} \quad \text{for } i = x, y, z
    \end{equation}
\end{enumerate}

\section*{Core Takeaway}
\textbf{The Uncertainty Principle:} We cannot simultaneously know the values of non-commuting observables.
\begin{itemize}
    \item Precise knowledge of $S_z$ ($\Delta S_z = 0$) forces maximum uncertainty in $S_x$ and $S_y$ ($\Delta S_{x,y} = \hbar/2$).
\end{itemize}

\end{document}